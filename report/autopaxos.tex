% !TEX TS-program = pdflatexmk
% For TeXShop on OS X and Herbert Schulz's latexmk engine.

\documentclass[pageno]{jpaper}
\usepackage{url}
\usepackage{titling}
\usepackage{amsmath}
\usepackage{listings}
\usepackage{enumerate}
\usepackage{mathtools}
\usepackage{amsthm}
\usepackage{lipsum}

\newcommand{\asplossubmissionnumber}{XXX}


\title{Autopaxos: Efficiently Achieving Consensus in a Changing Environment}
\author{Bob Adolf and Mike Rizzo\\Harvard University}
\date{}
%\setlength{\droptitle}{-50pt} %maketitle vskips 60pt, this takes most of it away
\pretitle{% The ASPLOS title, without the ASPLOS submission and extra vspace
\begin{center}%
\normalfont\Large\bfseries%
}
\posttitle{\par\end{center}}

\begin{document}

\maketitle

\begin{abstract}
Herein we describe the development of \emph{Autopaxos}, a distributed-consensus system that measures and adapts to environment.  Because a network environment continuously changes, Paxos parameters that were initially determined might be adversely affecting the system's performance once its environment sufficiently changes.  In this way, we hypothesized that a Paxos System that measures and adapts to its changing environment will perform better than one that does not.  We implemented our Paxos Servers to monitor the environment by tracking the network's latency through RTT estimates and node drop events.  We developed an analytical cost-model to optimize performance by picking the best heartbeat frequency and master timeout parameters for the current environment.  We found that \emph{Autopaxos} appropriately measures its environment and adjusts its parameters for the best policy metric score, but a difference in performance was not observable.
\end{abstract}

% !TEX root = autopaxos.tex
% !TEX TS-program = pdflatexmk
% For TeXShop on OS X and Herbert Schulz's latexmk engine.

\section{Introduction}
A common problem among distributed networks is achieving fault-tolerant consensus among the different members of a quorum of participants.  This is commonly achieved using the Paxos Algorithm which Lamport developed in order to solve the consensus problem \cite{Lamport01}.  There has been much attention in the literature about the application of Paxos in different situations other than simple consensus such as the use in distributed, fault-tolerant log management.  

It has been noted in multiple locations the difficulty of translating what Lamport deems the "simple algorithm" to code with the developers encountering numerous situations and special cases in the translation of the algorithm to a real-world, working system not described in the original paper \cite{Chandra07}.  This paper uses the Paxos Algorithm to implement a fault-tolerant distributed consensus system under the direction of \textit{Paxos Made Live}. 

Our Paxos system consisted of [X] lines of C++ code written using the event-driven C++ library Tamer \cite{Krohn07}.  Specifically, we implemented Multi-Paxos whereby the quorum of nodes elects a master for a length of time in the future (master timeout) such that only the master will serve requests for its duration as master.  The master also has the option of remaining master by sending out messages (called "heartbeats") to the other nodes in the quorum at a fixed interval.  Using this model, the other nodes in the quorum will elect a new master once the old one does not send a heartbeat which either may be due to the heartbeat not being sent in time or the master died.
% FIXME: should there be more here about the actual design of the system like Servers containing both acceptors and proposers?
% !TEX root = autopaxos.tex
% !TEX TS-program = pdflatexmk
% For TeXShop on OS X and Herbert Schulz's latexmk engine.

\section{Motivation}
System tuning is hard; some consider it an art, and many admins dedicate a are employed specifically to tweak keep large installations running.
Spending weeks or even months on finely crafting heuristics and parameters for a large installation can have measurable impacts, but these gains are fickle.
A failure, a new component, or a system upgrade can wipe out all that effort in moments.
Perhaps the system still functions correctly, even with reasonable performance, but that last slice is gone, and replacing it requires yet another sizable of time and effort.

Our goal is to automate these tasks through instrumentation, feedback, and search.
It is important to understand that we see this process as complementary to the design and evolution of the systems themselves.
A new protocol can enable new behaviors, unlock new application areas, and deliver qualitatively new services.
An adaptive parameter tuning framework does not replace this need; \emph{it makes them stronger}.

We turn this approach on the problem of efficiently achieving consensus in a distributed, evolving environment.
Specifically, we investigate a simple implementation of Paxos, similar to the Chubby system described by Burrows \cite{burrows2006chubby}.
Our model is a simple, unpipelined server with a long-lived master, little to no group management capability, and a structure otherwise very similar to Lamport's original description.

In our Paxos, liveness is provided through a somewhat complicated system of timeouts and heartbeat messages.
Often these constants are evaluated and set in the course of operation to determine the ideal values, and they are seldom revisited later.
It is these features that expose the tangle of tuning parameters and dependent behaviors which we aim to automatically tame.
Instead of forcing a human to derive and assign static values, we suggest a model where the system itself assigns its own parameters.
Specifically, we aim to build a Paxos system that measures its network and neighbor characteristics and responds appropriately by adjusting its timeout and RPC constants to achieve better performance.
We hypothesize that a Paxos system that \emph{measures} and \emph{adapts} to its environment will perform better than one that does not.


%The implementation of our Paxos contains certain constants that greatly affect the behavior of the system.  In our implementation, we have the two constants of "master timeouts" and "heartbeat frequencies" that can influence the behavior and performance of our system.  Often such constants are evaluated and determined in the network of operation to determine the ideal values for the best overall performance, and these values are seldom revisited at later times.  Because the conditions of network (i.e. network latency) can change [--citation?--], as time goes on, those initially determined constants may no longer be the best values given the change in the network.

%Because of this, we hypothesized that a Paxos system that \textit{measures} and \textit{adapts} to its environment will perform better than one that does not.  Specifically, we aimed to build a Paxos system that measured its network latency and node drops and responds appropriately by adjusting its master timeout and heartbeat frequency constants to better perform in its current environment.  
% !TEX root = autopaxos.tex
% !TEX TS-program = pdflatexmk
% For TeXShop on OS X and Herbert Schulz's latexmk engine.

\newcommand{\Tbf}{T_{\mathit{bf}}}
\newcommand{\Thb}{T_{\mathit{hb}}}
\newcommand{\Tto}{T_{\mathit{to}}}
\newcommand{\Tl}{T_{\mathit{l}}}
\newcommand{\Cr}{C_{\mathit{r}}}
\newcommand{\Chb}{C_{\mathit{hb}}}
\newcommand{\Pff}{P_{\mathit{ff}}}
\newcommand{\Ptf}{P_{\mathit{tf}}}

\section{Design}
We designed our model system around the same abstractions provided by Lamport in his original Paxos paper.
Our system contains modules for Paxos proposers and accepters, as well as client-level server objects.
While notionally separate entities, practically speaking the system can be viewed as a near-monolithic collection of coroutines written using the tamer \cite{Krohn07} event programming DSL.

We have not streamlined or optimized our implementation of Paxos.
As previously mentioned, our goal is not to demonstrate the fastest or most efficient version of a distributed consensus;
instead, we show that our work can improve the performance of an arbitrary algorithm without any special consideration at the protocol level.

%Our design of the system included the implementation of a \texttt{Paxos\_Server} class that contained within it the components of Paxos---the \texttt{Paxos\_Acceptors} and \texttt{Paxos\_Proposers}.  Each \texttt{Paxos\_Server} contained within it the different parameters to be set and used.  As development progressed, it became clear that the lines between each Paxos component (acceptor and proposer) and the server were blurring, so in future development, these distinctions will likely be removed.
In general, our approach can be broken down into three categories:
1) measurement: we instrument our Paxos code to capture information about its surrounding environment;
2) policy: we separate the concerns of making good decisions about how and what to do with the more mundane concerns of actually accomplishing them;
and 3) tuning: ultimately, any feedback loop needs an actuator for activity, and ours is no different.

\subsection{Measurement}
In order to capture the requisite statistics, we augmented our Paxos implementation with a centralized clearinghouse module we call the \texttt{Telemetry} class.
We are primarily interested in collected two salient pieces of information:
an estimate of the communications latency in the system, and the approximate failure rate of our nodes.

Our Paxos code uses a long-lived master, and heartbeat messages allow subordinates in the system to determine when the master has failed.
Since these messages are regularly sent to all nodes in the system, they are an ideal mechanism for measuring the point-to-point round trip time (RTT).
Similarly, node faults are detected through timeouts on expected messages, usually heartbeat messages and their responses.
We simply mark every timer expiration with code that records relevant information about the failure event to the \texttt{Telementry} class.

It is important to recognize that both of these features are \emph{estimates} of actual behavior.
The system has no oracle to guide it, and we study the accuracy of these methods in section~\ref{evaluation}.

%The \texttt{Telemetry} class has components to track the different measures of our environment of which we were interested including its latency and node failure rates.  In order to ensure that all members of the quorum maintain the same constants, static methods of the \texttt{Telemetry} class were used.

% latency
%\texttt{Telemetry} contains an average round-trip time (RTT) measure used to estimate the average network latency.  The current master calculates the average RTT from sending its heartbeat to all of the nodes in the quorum, then it updates the overall estimated RTT in Telemetry which is subsequently used to calculate the latency (RTT / 2). %FIXME: more details--update_rtt_estimate

% failure
%Failures are tracked in a similar way to the RTT: each time a node thinks a failure has occurred, it updates a statically global failure tracker in \texttt{Telemetry}.  This happens in two settings:
%\begin{enumerate}
%	\item The master notices a node's failure when sending a heartbeat from its response timing out.
%	\item A node notices a master's failure from a master's timeout.
%\end{enumerate}
%In either case, there can be either true failures and false failures, but nodes cannot tell the difference; both are thus recorded in \texttt{Telemetry}.  The distinction is made, though, between perceived master failures and perceived node failures (with the latter being a perceived master failure).

\subsection{Policy}
% policy = minimize cost
% explain cost models (uptime, traffic, recovery)
With more information, we enable informed decision-making.
Our policy module is responsible for converting the raw data from software instrumentation into actionable decisions.
Specifically, we use an analytical cost-model to optimize for performance.
At a high level, the model is just a mathematical way of describing ``goodness.''
While many others would have worked, we chose to express efficacy as the ratio of client availability (simplified as uptime) to cost (data transferred across the network).
To be concrete, we use the set of equations below to quantify what we mean by these terms:

\begin{equation}
	\mathrm{PMetric} = \frac%
{	\Tbf - \Ptf - \Pff } % Uptime
{	\frac{\Chb}{\Thb} + \frac{\Cr}{\Tbf} + \Pff\Cr } % Traffic
\end{equation}

where $\Tbf$ is the mean time between failures (MTBF), $\Tto$ is the master timeout constant, $\Thb$ is the heartbeat interval, $\Chb$ is the cost of sending a single pointwise heartbeat, $\Pff$ and $\Ptf$ are the probabilities of our timeout flagging a false failure or detecting a true failure, respectively, and $\Cr$ is the cost of recovery, that is, the cost of recovering from a master failure.
The subexpressions are $\Pff$ and $\Ptf$ are:

\begin{equation}
	\Pff = \mathrm{CDF}_{N(0,\sigma)} \Thb + \Tl - \Tto
\end{equation}

\begin{equation}
	\Ptf = 1-\mathrm{CDF}_{Exp(\frac{1}{\Tbf-\Tl})} 1000
\end{equation}

where $\Tl$ is the latency and the notation $\mathrm{CDF}_{N(\mu,\sigma)}$ and $\mathrm{CDF}_{\mathrm{Exp}(\lambda)}$ represent the cumulative distribution functions for a normal and exponential distributions.

\subsection{Tuning}
% adjust HB freq
Using this measure, each node in the quorum measures its environment adjusts its own parameters on regular intervals.
While we could have used more advanced techniques, our implementation uses brute force to identify the most viable parameters for a given system.
We simply iterate over the space of all possible parameter combinations until we arrive at a minimum.
We then directly assign values to various locations in memory.
No human intervention is necessary for this, and the process occurs many times per second.





% !TEX root = autopaxos.tex
% !TEX TS-program = pdflatexmk
% For TeXShop on OS X and Herbert Schulz's latexmk engine.

\section{Evaluation}
\lipsum[1-3]

%% !TEX root = autopaxos.tex
% !TEX TS-program = pdflatexmk
% For TeXShop on OS X and Herbert Schulz's latexmk engine.

\section{Related Work}
\lipsum[1-3]

% !TEX root = autopaxos.tex
% !TEX TS-program = pdflatexmk
% For TeXShop on OS X and Herbert Schulz's latexmk engine.

\section{Conclusions}
The reality of distributed networks is that the environment of the system is not static, so any initial parameters determined to improve performance for an implementation of Paxos must be revisited.  To overcome this, we designed a Paxos System that measures and adapts to its environment.  In this way, our Paxos Servers measure the network latency and node failure rate and duly adjust their heartbeat frequencies and master timeouts in order to perform better in the current environment's landscape.

We initially hypothesized that a Paxos System that appropriately tunes its parameters with the current environment would perform better than one that does not.  We found that indeed our Paxos Servers do measure their current environment and accordingly adjust their parameters to those best for the situation; however, such alterations did not have any noticeable effect on the overall performance of the system compared to a system that did not modify its parameters.  This is most likely due to the environmental conditions to which we subjected our system not being extreme enough for the benefits of our tuning parameters to be appreciably observable.  Further work will be needed in order to establish whether, in more severe environments, a Paxos System that autotunes its parameters in the way we have described will perform better than its non-autotuning counterpart.

\bibliographystyle{plain}
\bibliography{autopaxos}

\end{document}
