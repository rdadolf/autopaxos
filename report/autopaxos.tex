% !TEX TS-program = pdflatexmk
% For TeXShop on OS X and Herbert Schulz's latexmk engine.

\documentclass[pageno]{jpaper}
\usepackage{url}
\usepackage{titling}
\usepackage{amsmath}
\usepackage{listings}
\usepackage{enumerate}
\usepackage{mathtools}
\usepackage{amsthm}
\usepackage{lipsum}

\newcommand{\asplossubmissionnumber}{XXX}


\title{Autopaxos: Efficiently Achieving Consensus in a Changing Environment}
\author{Bob Adolf and Mike Rizzo\\Harvard University}
\date{}
%\setlength{\droptitle}{-50pt} %maketitle vskips 60pt, this takes most of it away
\pretitle{% The ASPLOS title, without the ASPLOS submission and extra vspace
\begin{center}%
\normalfont\Large\bfseries%
}
\posttitle{\par\end{center}}

\begin{document}

\maketitle

\begin{abstract}
Herein we describe the development of \emph{Autopaxos}, a distributed-consensus system that measures and adapts to environment.  Because a network environment continuously changes, Paxos parameters that were initially determined might be adversely affecting the system's performance once its environment sufficiently changes.  In this way, we hypothesized that a Paxos System that measures and adapts to its changing environment will perform better than one that does not.  We implemented our Paxos Servers to monitor the environment by tracking the network's latency through RTT estimates and node drop events.  We developed an analytical cost-model to optimize performance by picking the best heartbeat frequency and master timeout parameters for the current environment.  We found that \emph{Autopaxos} appropriately measures its environment and adjusts its parameters for the best policy metric score, but a difference in performance was not observable.
\end{abstract}

% !TEX root = autopaxos.tex
% !TEX TS-program = pdflatexmk
% For TeXShop on OS X and Herbert Schulz's latexmk engine.

\section{Introduction}
A common problem among distributed networks is achieving fault-tolerant consensus among the different members of a quorum of participants.  This is commonly achieved using the Paxos Algorithm which Lamport developed in order to solve the consensus problem [--citation--].  There has been much attention in the literature about the application of Paxos in different situations other than simple consensus such as the use in distributed, fault-tolerant log management.  

It has been noted in multiple locations the difficulty of translating what Lamport deems the "simple algorithm" to code with the developers encountering numerous situations and special cases in the translation of the algorithm to a real-world, working system not described in the original paper [--citation of paxos made live--].  This paper uses the Paxos Algorithm to implement a fault-tolerant distributed consensus system under the direction of \textit{Paxos Made Live}. 

Our Paxos system consisted of [X] lines of C++ code written using the event-driven C++ library Tamer[--citation--].  Specifically, we implemented Multi-Paxos [--citation--] whereby the quorum of nodes elects a master for a length of time in the future (master timeout) such that only the master will serve requests for its duration as master.  The master also has the option of remaining master by sending out messages (called "heartbeats") to the other nodes in the quorum at a fixed interval.  Using this model, the other nodes in the quorum will elect a new master once the old one does not send a heartbeat which either may be due to the heartbeat not being sent in time or the master died.
% FIXME: should there be more here about the actual design of the system like Servers containing both acceptors and proposers?
% !TEX root = autopaxos.tex
% !TEX TS-program = pdflatexmk
% For TeXShop on OS X and Herbert Schulz's latexmk engine.

\section{Motivation}
The implementation of our Paxos contains certain constants that greatly affect the behavior of the system.  In our implementation, we have the two constants of "master timeouts" and "heartbeat frequencies" that can influence the behavior and performance of our system.  Often such constants are evaluated and determined in the network of operation to determine the ideal values for the best overall performance, and these values are seldom revisited at later times.  Because the conditions of network (i.e. network latency) can change [--citation?--], as time goes on, those initially determined constants may no longer be the best values given the change in the network.

Because of this, we hypothesized that a Paxos system that \textit{measures} and \textit{adapts} to its environment will perform better than one that does not.  Specifically, we aimed to build a Paxos system that measured its network latency and node drops and responds appropriately by adjusting its master timeout and heartbeat frequency constants to better perform in its current environment.  
% !TEX root = autopaxos.tex
% !TEX TS-program = pdflatexmk
% For TeXShop on OS X and Herbert Schulz's latexmk engine.

\section{Design}
\lipsum[1-3]

% !TEX root = autopaxos.tex
% !TEX TS-program = pdflatexmk
% For TeXShop on OS X and Herbert Schulz's latexmk engine.

\section{Evaluation}
\lipsum[1-3]

%\input{related}
% !TEX root = autopaxos.tex
% !TEX TS-program = pdflatexmk
% For TeXShop on OS X and Herbert Schulz's latexmk engine.

\section{Conclusions}
The reality of distributed networks is that the environment of the system is not static, so any initial parameters determined to improve performance for an implementation of Paxos must be revisited and tweaked in time.  To overcome this, we designed a Paxos System that measures and adapts itself to its environment.  In this way, our Paxos Servers measure the network latency and node failure rate, and they duly adjust their heartbeat frequencies and master timeouts in order to perform better in the current environment's landscape.

We hypothesized that a Paxos System that appropriately tunes its parameters with the current environment would perform better than one that does not.  We found that indeed our Paxos Servers do measure their current environment and accordingly adjust their parameters to those best for the situation, and this causes them to have  a higher associated PMetric than their non-adjusting counterparts; however, such alterations display little additional benefit past the initial jump in PMetric value.  Further, with the exception of a small time period where latency was relatively large, there was not a significant different in how an autotuning system performed compared to one that did not.  These imply that our PMetric does not capture the most meaningful information for making adjustments to the parameters of the Paxos Servers, and the real benefits only manifest in particular, extreme situations.  Further work will thus be needed to develop a better PMetric and to establish whether a Paxos System that autotunes its parameters in the way we have described can perform significantly better than its non-autotuning counterpart.

\bibliographystyle{plain}
\bibliography{autopaxos}

\end{document}
