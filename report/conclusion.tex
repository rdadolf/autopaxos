% !TEX root = autopaxos.tex
% !TEX TS-program = pdflatexmk
% For TeXShop on OS X and Herbert Schulz's latexmk engine.

\section{Conclusions}
The reality of distributed networks is that the environment of the system is not static, so any initial parameters determined to improve performance for an implementation of Paxos must be revisited.  To overcome this, we designed a Paxos System that measures and adapts to its environment.  In this way, our Paxos Servers measure the network latency and node failure rate and duly adjust their heartbeat frequencies and master timeouts in order to perform better in the current environment's landscape.

We initially hypothesized that a Paxos System that appropriately tunes its parameters with the current environment would perform better than one that does not.  We found that indeed our Paxos Servers do measure their current environment and accordingly adjust their parameters to those best for the situation; however, such alterations did not have any noticeable effect on the overall performance of the system compared to a system that did not modify its parameters.  This is most likely due to the environmental conditions to which we subjected our system not being extreme enough for the benefits of our tuning parameters to be appreciably observable.  Further work will be needed in order to establish whether, in more severe environments, a Paxos System that autotunes its parameters in the way we have described will perform better than its non-autotuning counterpart.