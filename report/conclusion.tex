% !TEX root = autopaxos.tex
% !TEX TS-program = pdflatexmk
% For TeXShop on OS X and Herbert Schulz's latexmk engine.

\section{Conclusions}
The reality of distributed networks is that the environment of the system is not static, so any initial parameters determined to improve performance for an implementation of Paxos must be revisited and tweaked in time.  To overcome this, we designed a Paxos System that measures and adapts itself to its environment.  In this way, our Paxos Servers measure the network latency and node failure rate, and they duly adjust their heartbeat frequencies and master timeouts in order to perform better in the current environment's landscape.

We hypothesized that a Paxos System that appropriately tunes its parameters with the current environment would perform better than one that does not.  We found that indeed our Paxos Servers do measure their current environment and accordingly adjust their parameters to those best for the situation, and this causes them to have  a higher associated PMetric than their non-adjusting counterparts; however, such alterations display little additional benefit past the initial jump in PMetric value.  Further, with the exception of a small time period where latency was relatively large, there was not a significant different in how an autotuning system performed compared to one that did not.  These imply that our PMetric does not capture the most meaningful information for making adjustments to the parameters of the Paxos Servers, and the real benefits only manifest in particular, extreme situations.  Further work will thus be needed to develop a better PMetric and to establish whether a Paxos System that autotunes its parameters in the way we have described can perform significantly better than its non-autotuning counterpart.