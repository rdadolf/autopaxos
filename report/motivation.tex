% !TEX root = autopaxos.tex
% !TEX TS-program = pdflatexmk
% For TeXShop on OS X and Herbert Schulz's latexmk engine.

\section{Motivation}
System tuning is hard; indeed, some consider it an art, and many systems admins are often employed specifically to keep large installations running and tweak underlying parameters.
Spending weeks or even months on finely crafting heuristics and parameters for a large installation can have measurable impacts, but these gains are fickle.
A failure, a new component, or a system upgrade can wipe out all that effort in moments.
Perhaps the system still functions correctly, even with reasonable performance, but that last slice is gone, and replacing it requires yet another sizable of time and effort.

Our goal is to automate these tasks through instrumentation, feedback, and search.
It is important to understand that we see this process as complementary to the design and evolution of the systems themselves.
A new protocol can enable new behaviors, unlock new application areas, and deliver qualitatively new services.
An adaptive parameter tuning framework does not replace this need; \emph{it makes them stronger}.

We turn this approach on the problem of efficiently achieving consensus in a distributed, evolving environment.
Specifically, we investigate a simple implementation of Paxos, similar to the Chubby system described by Burrows \cite{burrows2006chubby}.
Our model is a simple, unpipelined server with a long-lived master, little to no group management capability, and a structure otherwise very similar to Lamport's original description.

In our Paxos, liveness is provided through a somewhat complicated system of timeouts and heartbeat messages.
Often these constants are evaluated and set in the course of operation to determine the ideal values, and they are seldom revisited later.
It is these features that expose the tangle of tuning parameters and dependent behaviors which we aim to automatically tame.
Instead of forcing a human to derive and assign static values, we suggest a model where the system itself assigns its own parameters.
Specifically, we aim to build a Paxos system that measures its network and neighbor characteristics and responds appropriately by adjusting its timeout and RPC constants to achieve better performance.
We hypothesize that a Paxos system that \emph{measures} and \emph{adapts} to its environment will perform better than one that does not.