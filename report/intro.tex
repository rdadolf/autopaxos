% !TEX root = autopaxos.tex
% !TEX TS-program = pdflatexmk
% For TeXShop on OS X and Herbert Schulz's latexmk engine.

\section{Introduction}
A common problem among distributed networks is achieving fault-tolerant consensus among the different members of a quorum of participants.  This is commonly achieved using the Paxos Algorithm which Lamport developed in order to solve the consensus problem \cite{Lamport01}.  There has been much attention in the literature about the application of Paxos in different situations other than simple consensus such as the use in distributed, fault-tolerant log management.  

It has been noted in multiple locations the difficulty of implementing what Lamport deems the "simple algorithm" to code.  Developers encounter numerous situations and special cases in the translation of the algorithm to a real-world, working system not described in the original paper \cite{Chandra07}.  This paper uses the Paxos Algorithm to implement a fault-tolerant distributed consensus system under the direction of \textit{Paxos Made Live}. 

Our Paxos system was implemented in C++ using the event-driven C++ library Tamer \cite{Krohn07}.  Specifically, we implemented Multi-Paxos whereby the quorum of nodes elects a master for a length of time in the future (master timeout) such that only the master will serve requests for its duration as master.  The master also has the option of remaining master by sending out messages (called "heartbeats") to the other nodes in the quorum at a fixed interval.  Using this model, the other nodes in the quorum will elect a new master once the old one times out which either may be due to the heartbeat not being sent in time or the master dying.